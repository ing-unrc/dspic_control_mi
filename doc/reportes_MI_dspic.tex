%%%%%%%%%%%%%%%%%%%%%%%%%%%%%%%%%%%%%%%%%%%%%%%%%%%%%%%%%%%%%%%%%%%%%%%%%%%%%%%%%
% REPORTE DE DESARROLLO DE ALGORITMOS PARA PIC
% Facundo Aguilera
% 04/2019
%
%
% Basado en plantilla: 
% Linux and Unix Users Group at Virginia Tech Wiki 
% (https://vtluug.org/wiki/Example_LaTeX_chem_lab_report)
% License: CC BY-NC-SA 3.0 (http://creativecommons.org/licenses/by-nc-sa/3.0/)
%
%%%%%%%%%%%%%%%%%%%%%%%%%%%%%%%%%%%%%%%%%%%%%%%%%%%%%%%%%%%%%%%%%%%%%%%%%%%%%%%%%

%----------------------------------------------------------------------------------------
%	PACKAGES AND DOCUMENT CONFIGURATIONS
%----------------------------------------------------------------------------------------

\documentclass{article}

\usepackage{siunitx} % Provides the \SI{}{} and \si{} command for typesetting SI units
\usepackage{graphicx} % Required for the inclusion of images
\usepackage{natbib} % Required to change bibliography style to APA
\usepackage{amsmath} % Required for some math elements 


% Ajuste de separación parrafos
\setlength{\parindent}{0em} % Sangría
\setlength{\parskip}{1em}   % Espaciado

\renewcommand{\labelenumi}{\alph{enumi}.} % Make numbering in the enumerate environment by letter rather than number (e.g. section 6)


%----------------------------------------------------------------------------------------
%	DOCUMENT INFORMATION
%----------------------------------------------------------------------------------------

\title{Desarrollo de control sin sensor de velocidad para PIC} % Title

\author{Facundo Aguilera} % Author name

\date{\today} % Date for the report

\begin{document}
	
	\maketitle 
	

	% If you wish to include an abstract, uncomment the lines below
	% \begin{abstract}
	% Abstract text
	% \end{abstract}
	
	%----------------------------------------------------------------------------------------
	%	SECTION 1
	%----------------------------------------------------------------------------------------
	
	\section{Objetivos}
	

	
	
	Obtener una estrategia sin sensor de vel que permita tener un par de arranque alto, de modo que un vehículo pueda acelerar sin caer al arrancarlo en una pendiente de subida.
	
	
	% If you have more than one objective, uncomment the below:
	\begin{description}
	\item[Objetivo 1] \hfill \\
	Desplazamiento del VEU menor a 20 cm.
	\item[Objetivo 2] \hfill \\
	Par constante en régimen luego del arranque, para un rango ampliado de velocidades.
	\end{description}
	

	
	%----------------------------------------------------------------------------------------
	%	TEST 1
	%----------------------------------------------------------------------------------------
	
	\section{Etapa 1}
	
	\begin{description}
		\item[Fecha]  04/30/2019 
		\item[Tema] Ajuste de las tensiones. Se observan valores de tensión distorsionadas al usar el control, por lo que se propone buscar formas de corregirlas.
	\end{description}


	
	
	\subsection{Etapa 1}
	
		Se observa que al poner tensiones perfectamente sinusoidales, las corrientes aparecen deformadas. Se ponen las salidas $v_\alpha$ y $v_\beta$ en SVGEN de MATLAB, Fig.~\ref{fig:bloque_svgen_matlab} \textsc{}y se verifica que hay algún problema en la generación de las señales de conmutación, Fig.~\ref{fig:comparacion_corrientes}.
	
	\begin{figure}
		\centering
		\includegraphics[width=0.3\linewidth]{figuras/0430_bloque_svgen_matlab}
		\caption[Ejemplo de bloque SVGEN de MATLAB.]{}
		\label{fig:bloque_svgen_matlab}
	\end{figure}
	
	
	
	\begin{figure}[h]
	\begin{center}
		\includegraphics[width=0.65\textwidth]{/home/facu/documentos/GEA/Desarrollos/control_MI/algoritmos_pic/dspic_control_mi/doc/figuras/0430_comparacion_corrientes_svgen.png} % Include the image placeholder.png
		\caption{Figure caption.}
		\label{fig:comparacion_corrientes}
	\end{center}
	\end{figure}
	
	%----------------------------------------------------------------------------------------
	%	SECTION 3
	%----------------------------------------------------------------------------------------
	
	\section{Sample Calculation}
	
	\begin{tabular}{ll}
		Mass of magnesium metal & = \SI{8.59}{\gram} - \SI{7.28}{\gram}\\
		& = \SI{1.31}{\gram}\\
		Mass of magnesium oxide & = \SI{9.46}{\gram} - \SI{7.28}{\gram}\\
		& = \SI{2.18}{\gram}\\
		Mass of oxygen & = \SI{2.18}{\gram} - \SI{1.31}{\gram}\\
		& = \SI{0.87}{\gram}
	\end{tabular}
	

	
	%----------------------------------------------------------------------------------------
	%	SECTION 4
	%----------------------------------------------------------------------------------------
	
	\section{Results and Conclusions}
	
	The atomic weight of magnesium is concluded to be \SI{24}{\gram\per\mol}, as determined by the stoichiometry of its chemical combination with oxygen. This result is in agreement with the accepted value.
	

	
	%----------------------------------------------------------------------------------------
	%	SECTION 5
	%----------------------------------------------------------------------------------------
	
	\section{Discussion of Experimental Uncertainty}
	
	The accepted value (periodic table) is \SI{24.3}{\gram\per\mole} \cite{Smith:2012qr}. The percentage discrepancy between the accepted value and the result obtained here is 1.3\%. Because only a single measurement was made, it is not possible to calculate an estimated standard deviation.
	
	The most obvious source of experimental uncertainty is the limited precision of the balance. Other potential sources of experimental uncertainty are: the reaction might not be complete; if not enough time was allowed for total oxidation, less than complete oxidation of the magnesium might have, in part, reacted with nitrogen in the air (incorrect reaction); the magnesium oxide might have absorbed water from the air, and thus weigh ``too much." Because the result obtained is close to the accepted value it is possible that some of these experimental uncertainties have fortuitously cancelled one another.
	
	%----------------------------------------------------------------------------------------
	%	SECTION 6
	%----------------------------------------------------------------------------------------
	
	\section{Answers to Definitions}
	
	\begin{enumerate}
		\begin{item}
			The \emph{atomic weight of an element} is the relative weight of one of its atoms compared to C-12 with a weight of 12.0000000$\ldots$, hydrogen with a weight of 1.008, to oxygen with a weight of 16.00. Atomic weight is also the average weight of all the atoms of that element as they occur in nature.
		\end{item}
		\begin{item}
			The \emph{units of atomic weight} are two-fold, with an identical numerical value. They are g/mole of atoms (or just g/mol) or amu/atom.
		\end{item}
		\begin{item}
			\emph{Percentage discrepancy} between an accepted (literature) value and an experimental value is
			\begin{equation*}
			\frac{\mathrm{experimental\;result} - \mathrm{accepted\;result}}{\mathrm{accepted\;result}}
			\end{equation*}
		\end{item}
	\end{enumerate}
	
	%----------------------------------------------------------------------------------------
	%	BIBLIOGRAPHY
	%----------------------------------------------------------------------------------------
	
	\bibliographystyle{apalike}
	
	\bibliography{sample}
	
	%----------------------------------------------------------------------------------------
	
	
\end{document}