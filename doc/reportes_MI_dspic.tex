%%%%%%%%%%%%%%%%%%%%%%%%%%%%%%%%%%%%%%%%%%%%%%%%%%%%%%%%%%%%%%%%%%%%%%%%%%%%%%%%%
% REPORTE DE DESARROLLO DE ALGORITMOS PARA PIC
% Facundo Aguilera
% 04/2019
%
%
% Basado en plantilla: 
% Linux and Unix Users Group at Virginia Tech Wiki 
% (https://vtluug.org/wiki/Example_LaTeX_chem_lab_report)
% License: CC BY-NC-SA 3.0 (http://creativecommons.org/licenses/by-nc-sa/3.0/)
%
%%%%%%%%%%%%%%%%%%%%%%%%%%%%%%%%%%%%%%%%%%%%%%%%%%%%%%%%%%%%%%%%%%%%%%%%%%%%%%%%%

%----------------------------------------------------------------------------------------
%	PACKAGES AND DOCUMENT CONFIGURATIONS
%----------------------------------------------------------------------------------------

\documentclass{article}

\usepackage{siunitx} % Provides the \SI{}{} and \si{} command for typesetting SI units
\usepackage{graphicx} % Required for the inclusion of images
\usepackage{natbib} % Required to change bibliography style to APA
\usepackage{amsmath} % Required for some math elements 


% Ajuste de separación parrafos
\setlength{\parindent}{0em} % Sangría
\setlength{\parskip}{1em}   % Espaciado

\renewcommand{\labelenumi}{\alph{enumi}.} % Make numbering in the enumerate environment by letter rather than number (e.g. section 6)


%----------------------------------------------------------------------------------------
%	DOCUMENT INFORMATION
%----------------------------------------------------------------------------------------

\title{Desarrollo de control sin sensor de velocidad para PIC} % Title

\author{Facundo Aguilera} % Author name

\date{\today} % Date for the report

\begin{document}
	
	\maketitle 
	

	% If you wish to include an abstract, uncomment the lines below
	% \begin{abstract}
	% Abstract text
	% \end{abstract}
	
	%----------------------------------------------------------------------------------------
	%	SECTION 1
	%----------------------------------------------------------------------------------------
	
	\section{Objetivos}
	

	
	
	Obtener una estrategia sin sensor de vel que permita tener un par de arranque alto, de modo que un vehículo pueda acelerar sin caer al arrancarlo en una pendiente de subida.
	
	
	% If you have more than one objective, uncomment the below:
	\begin{description}
	\item[Objetivo 1] \hfill \\
	Desplazamiento del VEU menor a 20 cm.
	\item[Objetivo 2] \hfill \\
	Par constante en régimen luego del arranque, para un rango ampliado de velocidades.
	\end{description}
	

	
	%----------------------------------------------------------------------------------------
	%	Problema de medición de tensiones
	%----------------------------------------------------------------------------------------
	
	\section{Medición de tensiones}
	
	
		\subsection{Reporte 1}
			\begin{description}
				\item[Fecha]  04/30/2019 
				\item[Tema] Ajuste de las tensiones. Se observan valores de tensión distorsionadas al usar el control, por lo que se propone buscar formas de corregirlas.
			\end{description}
		
			Se observa que al poner tensiones perfectamente sinusoidales, las corrientes aparecen deformadas. Se ponen las salidas $v_\alpha$ y $v_\beta$ en SVGEN de MATLAB, Fig.~\ref{fig:bloque_svgen_matlab}, perfectamente senoidales, Fig.~\ref{fig:tensiones_para_svgen},  y se verifica que hay algún problemas en la generación de las señales de conmutación a partir de notar una deformación en las señales de corriente, Fig.~\ref{fig:comparacion_corrientes}.
		
			\begin{figure}
				\centering
				\includegraphics[width=0.3\linewidth]{figuras/0430_bloque_svgen_matlab}
				\caption[]{Ejemplo de bloque SVGEN de MATLAB.}
				\label{fig:bloque_svgen_matlab}
			\end{figure}
			
			\begin{figure}
				\centering
				\includegraphics[width=0.7\linewidth]{figuras/0430_tensiones_para_svgen}
				\caption[]{Tensiones generadas para probar las corrientes.}
				\label{fig:tensiones_para_svgen}
			\end{figure}
		
			
			\begin{figure}[h]
			\begin{center}
				\includegraphics[width=0.65\textwidth]{/home/facu/documentos/GEA/Desarrollos/control_MI/algoritmos_pic/dspic_control_mi/doc/figuras/0430_comparacion_corrientes_svgen.png} % Include the image placeholder.png
				\caption{Corrientes con SVGEN de simulink (iz) y con svgen de pic (der).}
				\label{fig:comparacion_corrientes}
			\end{center}
			\end{figure}
		
			Se define usar en MATLAB el SVPWM propio, y en pic verificar en hardware para ver si tiene mismo comportamiento.
	
		\subsection{Etapa 2}
	

	
	%----------------------------------------------------------------------------------------
	%	SECTION 3
	%----------------------------------------------------------------------------------------
	\section{Estimación de velocidad}	
		
		\subsection{Reporte 1}
			\begin{description}
				\item[Fecha]  1/05/2019 
				\item[Tema] Estimación de velocidad del SMO presenta problemas al frenar, se buscan forma de mejorarlo
			\end{description}
		
			En base a ajuste de ganancias del observador de Zaky, se observan las siguientes características:
			
			\begin{itemize}
				\item $\delta$ Suaviza oscilaciones de la velocidad
				\item $gama$ Es necesario subirlo bastante, aunque valores muy altos comienzan a incremenatar el error de vel y las oscilaciones
				\item $K3,\,K4$ Valores bajos no permiten que la velocidad converja, valores altos rompen el funcionamiento del observador.
				\item Por lo general no se pueden eliminar las oscilaciones y conseguir que funcione aceptablemente durante el frenado. 
			\end{itemize}
	
	%----------------------------------------------------------------------------------------
	%	BIBLIOGRAPHY
	%----------------------------------------------------------------------------------------
	
	\bibliographystyle{apalike}
	
	\bibliography{sample}
	
	%----------------------------------------------------------------------------------------
	
	
\end{document}