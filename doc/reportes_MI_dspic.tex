%%%%%%%%%%%%%%%%%%%%%%%%%%%%%%%%%%%%%%%%%%%%%%%%%%%%%%%%%%%%%%%%%%%%%%%%%%%%%%%%%
% REPORTE DE DESARROLLO DE ALGORITMOS PARA PIC
% Facundo Aguilera
% 04/2019
%
%
% Basado en plantilla: 
% Linux and Unix Users Group at Virginia Tech Wiki 
% (https://vtluug.org/wiki/Example_LaTeX_chem_lab_report)
% License: CC BY-NC-SA 3.0 (http://creativecommons.org/licenses/by-nc-sa/3.0/)
%
%%%%%%%%%%%%%%%%%%%%%%%%%%%%%%%%%%%%%%%%%%%%%%%%%%%%%%%%%%%%%%%%%%%%%%%%%%%%%%%%%

%----------------------------------------------------------------------------------------
%	PACKAGES AND DOCUMENT CONFIGURATIONS
%----------------------------------------------------------------------------------------

\documentclass{article}

\usepackage{siunitx} % Provides the \SI{}{} and \si{} command for typesetting SI units
\usepackage{graphicx} % Required for the inclusion of images
\usepackage{natbib} % Required to change bibliography style to APA
\usepackage{amsmath} % Required for some math elements 


\usepackage{listings}

\usepackage{hyperref}

% Ajuste de separación parrafos
\setlength{\parindent}{0em} % Sangría
\setlength{\parskip}{1em}   % Espaciado

\renewcommand{\labelenumi}{\alph{enumi}.} % Make numbering in the enumerate environment by letter rather than number (e.g. section 6)


%----------------------------------------------------------------------------------------
%	DOCUMENT INFORMATION
%----------------------------------------------------------------------------------------

\title{Desarrollo de control sin sensor de velocidad para PIC} % Title

\author{Facundo Aguilera} % Author name

\date{\today} % Date for the report

\begin{document}
	
	\maketitle 
	

	% If you wish to include an abstract, uncomment the lines below
	% \begin{abstract}
	% Abstract text
	% \end{abstract}
	
	%----------------------------------------------------------------------------------------
	%	SECTION 1
	%----------------------------------------------------------------------------------------
	
	\section{Objetivos}
	

	
	
	Obtener una estrategia sin sensor de vel que permita tener un par de arranque alto, de modo que un vehículo pueda acelerar sin caer al arrancarlo en una pendiente de subida.
	
	
	% If you have more than one objective, uncomment the below:
	\begin{description}
	\item[Objetivo 1] \hfill \\
	Desplazamiento del VEU menor a 20 cm.
	\item[Objetivo 2] \hfill \\
	Par constante en régimen luego del arranque, para un rango ampliado de velocidades.
	\end{description}
	

	
	%----------------------------------------------------------------------------------------
	%	Problema de medición de tensiones
	%----------------------------------------------------------------------------------------
	
	\section{Medición de tensiones}
	
	
		\subsection{Reporte 1}
			\begin{description}
				\item[Fecha]  04/30/2019 
				\item[Tema] Ajuste de las tensiones. Se observan valores de tensión distorsionadas al usar el control, por lo que se propone buscar formas de corregirlas.
			\end{description}
		
			Se observa que al poner tensiones perfectamente sinusoidales, las corrientes aparecen deformadas. Se ponen las salidas $v_\alpha$ y $v_\beta$ en SVGEN de MATLAB, Fig.~\ref{fig:bloque_svgen_matlab}, perfectamente senoidales, Fig.~\ref{fig:tensiones_para_svgen},  y se verifica que hay algún problemas en la generación de las señales de conmutación a partir de notar una deformación en las señales de corriente, Fig.~\ref{fig:comparacion_corrientes}.
		
			\begin{figure}
				\centering
				\includegraphics[width=0.3\linewidth]{figuras/0430_bloque_svgen_matlab}
				\caption[]{Ejemplo de bloque SVGEN de MATLAB.}
				\label{fig:bloque_svgen_matlab}
			\end{figure}
			
			\begin{figure}
				\centering
				\includegraphics[width=0.7\linewidth]{figuras/0430_tensiones_para_svgen}
				\caption[]{Tensiones generadas para probar las corrientes.}
				\label{fig:tensiones_para_svgen}
			\end{figure}
		
			
			\begin{figure}[h]
			\begin{center}
				\includegraphics[width=0.65\textwidth]{/home/facu/documentos/GEA/Desarrollos/control_MI/algoritmos_pic/dspic_control_mi/doc/figuras/0430_comparacion_corrientes_svgen.png} % Include the image placeholder.png
				\caption{Corrientes con SVGEN de simulink (iz) y con svgen de pic (der).}
				\label{fig:comparacion_corrientes}
			\end{center}
			\end{figure}
		
			Se define usar en MATLAB el SVPWM propio, y en pic verificar en hardware para ver si tiene mismo comportamiento.
	
		\subsection{Etapa 2}
	

	
	%----------------------------------------------------------------------------------------
	%	SECTION 3
	%----------------------------------------------------------------------------------------
	\section{Estimación de velocidad}	
		
		\subsection{Reporte 1}
			\begin{description}
				\item[Fecha]  1/05/2019 
				\item[Tema] Estimación de velocidad del SMO presenta problemas al frenar, se buscan forma de mejorarlo
			\end{description}
		
			En base a ajuste de ganancias del observador de Zaky, se observan las siguientes características:
			
			\begin{itemize}
				\item $\delta$ Suaviza oscilaciones de la velocidad
				\item $gama$ Es necesario subirlo bastante, aunque valores muy altos comienzan a incremenatar el error de vel y las oscilaciones
				\item $K3,\,K4$ Valores bajos no permiten que la velocidad converja, valores altos rompen el funcionamiento del observador.
				\item Por lo general no se pueden eliminar las oscilaciones y conseguir que funcione aceptablemente durante el frenado. 
			\end{itemize}
	
	
	\section{Estimación de flujo con PLL\_PIC}
	
	
		\subsection{Reporte 1}
			\begin{description}
				\item[Fecha]  07/05/2019 
				\item[Tema] Análisis de señales luego de mejora de formas de onda de la tensión.
			\end{description}
		
			Se observa:
			\begin{itemize}
				\item Los valores de fem tienen mucho menos ruido, se logra obtener par máximo hasta velocidades menores.
				\item Con la configuración actual no se logra arrancar en pendiente con el modelo de VEU.
			\end{itemize}
	
	\section{Estimación de flujo con PLL\_BIER}
	
		
		\subsection{Reporte 1}
		\begin{description}
			\item[Fecha]  08/05/2019 
			\item[Tema] Tratando de lograr el correcto funcionamiento
		\end{description}
	
	
		El signo de las corrientes obtenidas en $qd$ no coincide con el del modelo usado en el PIC. Ambas corrientes se tuvieron que usar con signo contrario, ver Fig.~\ref{fig:esquemapaper} y Código~\ref{cod:funcion}.
		
		\begin{figure}
			\centering
			\includegraphics[width=0.7\linewidth]{figuras/0508_esquema_paper}
			\caption{Esquema propuesto en el paper}
			\label{fig:esquemapaper}
		\end{figure}
		
		\begin{lstlisting}[language=MATLAB,caption={Contenido de función de MATLAB},label=cod:funcion,]
Vd = Valpha * cos(tita) + Vbeta  * sin(tita);
Vq = Vbeta  * cos(tita) - Valpha * sin(tita);
          
error = (  Vd - omega_est*sigma*Ls*i_q_ref + Rs*i_d_ref );
error = -( (omega_est<0)*2 -1)* error ;		
		\end{lstlisting}
  		
  		
  		\subsection{Reporte 2}
  		\begin{description}
  			\item[Fecha]  08/05/2019 
  			\item[Tema] Ajuste de PI
  		\end{description}
		
		Prueba con I y sin I en el PI.
		
		
		
    \section{Estimación de flujo con PLL\_PIC}
		
		
		\subsection{Reporte 1}
		\begin{description}
			\item[Fecha]  07/05/2019 
			\item[Tema] Análisis de señales luego de mejora de formas de onda de la tensión.
		\end{description}
		
		Se observa:
		\begin{itemize}
			\item Los valores de fem tienen mucho menos ruido, se logra obtener par máximo hasta velocidades menores.
			\item Con la configuración actual no se logra arrancar en pendiente con el modelo de VEU.
		\end{itemize}
		
	\section{Cuantificación observador Zaki}
		
        
        \subsection{Reporte: Cuantifiación en MATLAB}
        \begin{description}
        	\item[Fecha]  04/06/2019 
        \end{description}
		
        En implementaciones de TI, usan bastante la
        %\href{http://www.latex-project.org/}{latex project}
        \href{https://en.wikipedia.org/wiki/Trapezoidal_rule_(differential_equations)}{regla Trapezoidal} (aparte de Euler) para hacer la discretización. Por esto, se arman las etapas para habilitar la posibilidad de implementarlo con trapezoidal o Euler.

        Analizando nuevamente la implementación trapezoidal, se observa que es un método implícito, lo que requiere resolver muchas cosas en cada paso de integración. Se deja la posibilidad entonces de implementar un método Heun o similar explícito, aunque Euler en principio está funcionando.
        
        \subsubsection{Tiempo continuo}

            Ecuaciones en tiempo continuo:
            \begin{align*}
               \dot{\hat{\lambda}}_\alpha &=  \frac{L_M}{T_r} \hat{i}_\alpha 
                    - \frac{1}{T_r} \hat{\lambda}_\alpha  
                    - \hat{\omega} \hat{\lambda}_\beta 
                    + k_1\nu_\alpha + k_2\nu_\beta  \\
                \dot{\hat{\lambda}}_\beta &=  \frac{L_M}{T_r} \hat{i}_\beta 
                    - \frac{1}{T_r} \hat{\lambda}_\beta  
                    + \hat{\omega} \hat{\lambda}_\alpha 
                    - k_2\nu_\alpha + k_1\nu_\beta  \\
                \dot{\hat{i}}_\beta &=  -a \hat{i}_\beta 
                    + \frac{L_M}{b T_r} \hat{\lambda}_\beta  
                    - \frac{L_M}{b} \hat{\omega} \hat{\lambda}_\alpha 
                    + \frac{L_r}{b} v_\beta 
                    - \nu_\beta \\
                \dot{\hat{i}}_\alpha &=  -a \hat{i}_\alpha 
                    + \frac{L_M}{b T_r} \hat{\lambda}_\alpha  
                    + \frac{L_M}{b} \hat{\omega} \hat{\lambda}_\beta 
                    + \frac{L_r}{b} v_\alpha
                    - \nu_\alpha \\
                \dot{\hat{\omega}} &=  -\gamma \left( l_\alpha \hat{\lambda}_\beta 
                    - l_\beta \hat{\lambda}_\alpha  \right)
            \end{align*}
        


        \subsubsection{Tiempo discreto}
            En tiempo discreto, se define
            
            \begin{align*}
                f_{\lambda\alpha} &=  \frac{L_M}{T_r} \hat{i}_\alpha 
                    - \frac{1}{T_r} \hat{\lambda}_\alpha  
                    - \hat{\omega} \hat{\lambda}_\beta 
                    + k_1\nu_\alpha + k_2\nu_\beta  \\
                f_{\lambda\beta} &=  \frac{L_M}{T_r} \hat{i}_\beta 
                    - \frac{1}{T_r} \hat{\lambda}_\beta  
                    + \hat{\omega} \hat{\lambda}_\alpha 
                    - k_2\nu_\alpha + k_1\nu_\beta  \\
                f_{i\beta} &=  -a \hat{i}_\beta 
                    + \frac{L_M}{b T_r} \hat{\lambda}_\beta  
                    - \frac{L_M}{b} \hat{\omega} \hat{\lambda}_\alpha 
                    + \frac{L_r}{b} v_\beta 
                    - \nu_\beta \\
                f_{i\alpha} &=  -a \hat{i}_\alpha 
                    + \frac{L_M}{b T_r} \hat{\lambda}_\alpha  
                    + \frac{L_M}{b} \hat{\omega} \hat{\lambda}_\beta 
                    + \frac{L_r}{b} v_\alpha
                    - \nu_\alpha \\
                f_\omega &=  -\gamma \left( l_\alpha \hat{\lambda}_\beta 
                    - l_\beta \hat{\lambda}_\alpha  \right)                
            \end{align*}
            
            
            Para implementación Euler:
            
            \begin{equation}
                \begin{aligned}
                    \hat{\lambda}_\alpha(n) &= \hat{\lambda}_\alpha(n-1)
                        + T_{sc}  f_{\lambda\alpha}(n-1)  \\
                    \hat{\lambda}_\beta(n) &= \hat{\lambda}_\beta(n-1)
                        + T_{sc}  f_{\lambda\beta}(n-1) \\
                    \hat{i}_\beta(n) &= \hat{i}_\beta(n-1)
                        + T_{sc}  f_{i\beta}(n-1) \\
                    \hat{i}_\alpha(n) &= \hat{i}_\alpha(n-1) 
                        + T_{sc}  f_{i\alpha}(n-1) \\
                    \hat{\omega}(n) &= \hat{\omega}(n-1)
                        + T_{sc}  f_{\omega}(n-1)  
                \end{aligned}
            \end{equation}

        \subsubsection{Tiempo discreto en p.u.}
            Se define:
            \begin{align}
                \lambda_0 &= L_M I_0 \\
                I_0 &= \frac{\lambda_0}{L_M} 
            \end{align}

            Entonces las ecuaciones en p.u. pueden escribirse dividiendo por $\lambda_0$ y por $I_0$  según cada caso:
            
            \begin{equation}
                \begin{aligned}
                    f_{\lambda\alpha}^{pu} &=  \frac{L_M}{T_r} \frac{1}{L_M I_0} \hat{i}_\alpha 
                        - \frac{1}{T_r} \hat{\lambda}_\alpha^{pu}  
                        - \hat{\omega} \hat{\lambda}_\beta^{pu} 
                        + \frac{1}{\lambda_0} k_1\nu_\alpha + \frac{1}{\lambda_0} k_2\nu_\beta  \\
                    f_{\lambda\beta}^{pu} &=  \frac{L_M}{T_r} \frac{1}{L_M I_0} \hat{i}_\beta 
                        - \frac{1}{T_r} \hat{\lambda}_\beta^{pu}  
                        + \hat{\omega} \hat{\lambda}_\alpha^{pu} 
                        - \frac{1}{\lambda_0} k_2\nu_\alpha + \frac{1}{\lambda_0} k_1\nu_\beta  \\
                    f_{i\beta}^{pu} &=  -a  \hat{i}_\beta^{pu} 
                        + \frac{L_M}{b T_r} \frac{L_M}{\lambda_0} \hat{\lambda}_\beta  
                        - \frac{L_M}{b} \frac{L_M}{\lambda_0} \hat{\omega} \hat{\lambda}_\alpha 
                        + \frac{L_r}{b} \frac{1}{I_0} v_\beta 
                        - \frac{1}{I_0} \nu_\beta \\
                    f_{i\alpha}^{pu} &=  -a  \hat{i}_\alpha^{pu} 
                        + \frac{L_M}{b T_r} \frac{L_M}{\lambda_0} \hat{\lambda}_\alpha  
                        + \frac{L_M}{b} \frac{L_M}{\lambda_0} \hat{\omega} \hat{\lambda}_\beta 
                        + \frac{L_r}{b} \frac{1}{I_0} v_\alpha
                        - \frac{1}{I_0} \nu_\alpha \\
                    f_\omega^{pu} &=  -\gamma \frac{1}{\omega_0} \left( l_\alpha \hat{\lambda}_\beta 
                        - l_\beta \hat{\lambda}_\alpha  \right)
                \end{aligned}
            \end{equation}
            
            Operando:
            
            \begin{equation}
                \begin{aligned}
                    f_{\lambda\alpha}^{pu} &=  \frac{1}{T_r}  \hat{i}_\alpha^{pu}
                        - \frac{1}{T_r} \hat{\lambda}_\alpha^{pu}  
                        - \hat{\omega} \hat{\lambda}_\beta^{pu} 
                        + \frac{1}{\lambda_0} k_1\nu_\alpha + \frac{1}{\lambda_0} k_2\nu_\beta  \\
                    f_{\lambda\beta}^{pu} &=  \frac{1}{T_r}  \hat{i}_\beta^{pu} 
                        - \frac{1}{T_r} \hat{\lambda}_\beta^{pu}  
                        + \hat{\omega} \hat{\lambda}_\alpha^{pu} 
                        - \frac{1}{\lambda_0} k_2\nu_\alpha + \frac{1}{\lambda_0} k_1\nu_\beta  \\
                    f_{i\beta}^{pu} &=  -a  \hat{i}_\beta^{pu} 
                        + \frac{L_M^2}{b T_r}  \hat{\lambda}_\beta^{pu}  
                        - \frac{L_M^2}{b}  \hat{\omega} \hat{\lambda}_\alpha^{pu}
                        + \frac{L_r}{b I_0} v_\beta 
                        - \frac{1}{I_0} \nu_\beta \\
                    f_{i\alpha}^{pu} &=  -a  \hat{i}_\alpha^{pu} 
                        + \frac{L_M^2}{b T_r}  \hat{\lambda}_\alpha^{pu}  
                        +  \frac{L_M^2}{b} \hat{\omega} \hat{\lambda}_\beta^{pu}
                        + \frac{L_r}{b I_0} v_\alpha
                        - \frac{1}{I_0} \nu_\alpha \\
                    f_\omega^{pu} &=  -\gamma \frac{1}{\omega_0} \left( l_\alpha \hat{\lambda}_\beta 
                        - l_\beta \hat{\lambda}_\alpha  \right)
                \end{aligned}
            \end{equation}
            
            Pasando las variables restantes a PU:
            \begin{equation}
                \begin{aligned}
                    f_{\lambda\alpha}^{pu} &=  \frac{1}{T_r}  \hat{i}_\alpha^{pu}
                        - \frac{1}{T_r} \hat{\lambda}_\alpha^{pu}  
                        - \omega_0 \hat{\omega}^{pu} \hat{\lambda}_\beta^{pu} 
                        + \frac{1}{\lambda_0} k_1\nu_\alpha + \frac{1}{\lambda_0} k_2\nu_\beta  \\
                    f_{\lambda\beta}^{pu} &=  \frac{1}{T_r}  \hat{i}_\beta^{pu} 
                        - \frac{1}{T_r} \hat{\lambda}_\beta^{pu}  
                        + \omega_0 \hat{\omega}^{pu} \hat{\lambda}_\alpha^{pu} 
                        - \frac{1}{\lambda_0} k_2\nu_\alpha + \frac{1}{\lambda_0} k_1\nu_\beta  \\
                    f_{i\beta}^{pu} &=  -a  \hat{i}_\beta^{pu} 
                        + \frac{L_M^2}{b T_r}  \hat{\lambda}_\beta^{pu}  
                        - \frac{L_M^2}{b} \omega_0   \hat{\omega}^{pu} \hat{\lambda}_\alpha^{pu}
                        + \frac{L_r V_0}{b I_0} v_\beta^{pu} 
                        - \frac{1}{I_0} \nu_\beta \\
                    f_{i\alpha}^{pu} &=  -a  \hat{i}_\alpha^{pu} 
                        + \frac{L_M^2}{b T_r}  \hat{\lambda}_\alpha^{pu}  
                        +  \frac{L_M^2}{b} \omega_0  \hat{\omega}^{pu} \hat{\lambda}_\beta^{pu}
                        + \frac{L_r V_0}{b I_0} v_\alpha^{pu}
                        - \frac{1}{I_0} \nu_\alpha \\
                    f_\omega^{pu} &=  -\gamma \frac{\lambda_0}{\omega_0} 
                        \left( 
                            l_\alpha \hat{\lambda}_\beta^{pu} 
                            - l_\beta \hat{\lambda}_\alpha^{pu}  
                        \right) \\
                    \hat{\lambda}_\alpha^{pu}(n) &= \hat{\lambda}_\alpha^{pu}(n-1)
                        + T_{sc}  f_{\lambda\alpha}^{pu}(n-1)  \\
                    \hat{\lambda}_\beta(n) &= \hat{\lambda}_\beta^{pu}(n-1)
                        + T_{sc}  f_{\lambda\beta}^{pu}(n-1) \\
                    \hat{i}_\beta^{pu}(n) &= \hat{i}_\beta^{pu}(n-1)
                        + T_{sc}  f_{i\beta}^{pu}(n-1) \\
                    \hat{i}_\alpha^{pu}(n) &= \hat{i}_\alpha^{pu}(n-1) 
                        + T_{sc}  f_{i\alpha}^{pu}(n-1) \\
                    \hat{\omega}^{pu}(n) &= \hat{\omega}^{pu}(n-1)
                        + T_{sc}  f_{\omega}^{pu}(n-1)  
                \end{aligned}
            \end{equation}

        \subsubsection{Tiempo discreto con constantes}
            Usando IQ(15), se deben tener en cuenta los desplazamientos en las multiplicaciones:
            \begin{equation}
                \begin{aligned}
                    f_{\lambda\alpha}^{pu} &=  \frac{1}{T_r} 2^{15} \hat{i}_\alpha^{pu}
                        - \frac{1}{T_r}2^{15} \hat{\lambda}_\alpha^{pu}  
                        - \omega_0 2^{15} \hat{\omega}^{pu} \hat{\lambda}_\beta^{pu} 
                        + \frac{1}{\lambda_0} k_1\nu_\alpha + \frac{1}{\lambda_0} k_2\nu_\beta  \\
                    f_{\lambda\beta}^{pu} &=  \frac{1}{T_r} 2^{15} \hat{i}_\beta^{pu} 
                        - \frac{1}{T_r}2^{15} \hat{\lambda}_\beta^{pu}  
                        + \omega_0 2^{15} \hat{\omega}^{pu} \hat{\lambda}_\alpha^{pu} 
                        - \frac{1}{\lambda_0} k_2\nu_\alpha + \frac{1}{\lambda_0} k_1\nu_\beta  \\
                    f_{i\beta}^{pu} &=  -a 2^{15} \hat{i}_\beta^{pu} 
                        + \frac{L_M^2}{b T_r}2^{15}  \hat{\lambda}_\beta^{pu}  
                        - \frac{L_M^2}{b} \omega_0 2^{15}    \hat{\omega}^{pu} \hat{\lambda}_\alpha^{pu}
                        + \frac{L_r V_0}{b I_0}2^{15} v_\beta^{pu} 
                        - \frac{1}{I_0} \nu_\beta \\
                    f_{i\alpha}^{pu} &=  -a 2^{15} \hat{i}_\alpha^{pu} 
                        + \frac{L_M^2}{b T_r}2^{15}  \hat{\lambda}_\alpha^{pu}  
                        +  \frac{L_M^2}{b} \omega_0 2^{15}   \hat{\omega}^{pu} \hat{\lambda}_\beta^{pu}
                        + \frac{L_r V_0}{b I_0}2^{15} v_\alpha^{pu}
                        - \frac{1}{I_0} \nu_\alpha \\
                    f_\omega^{pu} &=  -\gamma \frac{\lambda_0}{\omega_0}2^{15}
                        \left( 
                            l_\alpha \hat{\lambda}_\beta^{pu} 
                            - l_\beta \hat{\lambda}_\alpha^{pu}  
                        \right) \\
                    \hat{\lambda}_\alpha^{pu}(n) &= \hat{\lambda}_\alpha^{pu}(n-1)
                        + T_{sc} 2^{15} f_{\lambda\alpha}^{pu}(n-1)  \\
                    \hat{\lambda}_\beta(n) &= \hat{\lambda}_\beta^{pu}(n-1)
                        + T_{sc} 2^{15} f_{\lambda\beta}^{pu}(n-1) \\
                    \hat{i}_\beta^{pu}(n) &= \hat{i}_\beta^{pu}(n-1)
                        + T_{sc} 2^{15} f_{i\beta}^{pu}(n-1) \\
                    \hat{i}_\alpha^{pu}(n) &= \hat{i}_\alpha^{pu}(n-1) 
                        + T_{sc} 2^{15} f_{i\alpha}^{pu}(n-1) \\
                    \hat{\omega}^{pu}(n) &= \hat{\omega}^{pu}(n-1)
                        + T_{sc} 2^{15} f_{\omega}^{pu}(n-1)  
                \end{aligned}
            \end{equation}

    
        Redefiniendo las constantes

        \begin{equation}
            \begin{aligned}
                f_{\lambda\alpha}^{pu} =& \left( K_{InvTr}   \hat{i}_\alpha^{pu} \right)>>15 
                    -  \left(K_{InvTr} \hat{\lambda}_\alpha^{pu}  \right)>>15 
                    - \left( K_{\omega} \hat{\omega}^{pu} \hat{\lambda}_\beta^{pu} \right)>>15 \\
                   & +  \left( K_{k1\nu} \nu_\alpha^{pu} \right)>>15 
                    +  \left( K_{k2\nu}\nu_\beta^{pu} \right) >>15\\
                f_{\lambda\beta}^{pu} =& \left(  K_{InvTr} \hat{i}_\beta^{pu} \right) >>15 
                    -  \left( K_{InvTr} \hat{\lambda}_\beta^{pu}  \right) >>15 
                    + \left( K_{\omega} \hat{\omega}^{pu} \hat{\lambda}_\alpha^{pu} \right) >>15 \\
                   & - \left( K_{k2\nu} \nu_\alpha^{pu} \right) >>15 
                    + \left( K_{k1\nu} \nu_\beta^{pu} \right) >>15 \\
                f_{i\beta}^{pu} = &  
                    \left(K_a \hat{i}_\beta^{pu}\right)  >>15   
                    + \left( K_{L_M\_bT_r}  \hat{\lambda}_\beta^{pu}  \right)  >>15 
                     - \left( K_{L_M\_b}  \hat{\omega}^{pu}  \hat{\lambda}_\alpha^{pu}\right)  >>15 \\
                    & + \left( K_{L_r\_b}  v_\beta^{pu} \right) >>15 
                     -    \nu_\beta^{pu}   \\
                f_{i\alpha}^{pu} = & \left( K_a \hat{i}_\alpha^{pu}        \right) >>15  
                   + \left( K_{L_M\_bT_r} \hat{\lambda}_\alpha^{pu}  \right)  >>15
                   + \left(  K_{L_M\_b} \hat{\omega}^{pu} \hat{\lambda}_\beta^{pu}   \right)>>15  \\
                   & + \left( K_{L_r\_b} v_\alpha^{pu} \right) >>15 
                   -  \nu_\alpha^{pu}  \\
                f_\omega^{pu} = & K_\gamma 
                    \left( 
                        l_\alpha \hat{\lambda}_\beta^{pu} 
                        - l_\beta \hat{\lambda}_\alpha^{pu}  
                    \right) 
            \end{aligned}
        \end{equation}
        
        Las constantes se definen como:
        \begin{equation}
            \begin{aligned}
                K_{InvTr}       &= \frac{1}{T_r} 2^{15}           &
                K_{k1\nu}       &= k_1 2^{15} \\
                K_{k2\nu}       &= k_2 2^{15} &
                K_{\omega}      &= \omega_0 2^{15}                 \\
                K_a             &=    -a 2^{15}                    &
                K_{L_M\_bT_r}   &= \frac{L_M^2}{b T_r}2^{15}      \\
                 K_{L_M\_b}     &=  \frac{L_M^2}{b}2^{15}                   &
                K_{L_r\_b}      &=  \frac{L_r V_0}{b I_0}2^{15}             \\
                \nu_\beta^{pu}  &= \frac{1}{I_0} \nu_\beta                  &
                \nu_\alpha^{pu} &= \frac{1}{I_0} \nu_\alpha                 \\
                 K_\gamma       &= -\gamma \frac{\lambda_0}{\omega_0}2^{15} &
                K_{T_s}         &= T_{sc} 2^{15}  
            \end{aligned}
        \end{equation}

        La ecuación en diferencias queda como: 
        \begin{equation}
            \begin{aligned}
                \hat{\lambda}_\alpha^{pu}(n) &= \hat{\lambda}_\alpha^{pu}\left(n-1\right)
                    + \left( K_{T_s}  f_{\lambda\alpha}^{pu}\left(n-1\right) \right) >> 15 \\
                \hat{\lambda}_\beta(n) &= \hat{\lambda}_\beta^{pu}\left(n-1\right)
                    + \left(  K_{T_s} f_{\lambda\beta}^{pu} \left(n-1\right) \right) >> 15 \\
                \hat{i}_\beta^{pu}(n) &= \hat{i}_\beta^{pu} \left(n-1\right)
                    +  \left(  K_{T_s}  f_{i\beta}^{pu} \left(n-1\right) \right) >> 15 \\
                \hat{i}_\alpha^{pu}(n) &= \hat{i}_\alpha^{pu} \left(n-1\right) 
                    +  \left( K_{T_s} f_{i\alpha}^{pu} \left(n-1\right) \right) >> 15 \\
                \hat{\omega}^{pu}(n) &= \hat{\omega}^{pu} \left(n-1\right)
                    +  \left(  K_{T_s}  f_{\omega}^{pu}\left(n-1\right)  \right) >> 15
            \end{aligned}
        \end{equation}


        Considerando que puede ocurrir que sea necesario hacer una pre-división a las constantes, en caso de que se pasen de rango, se definen constantes para los desplazamientos, y se redefinen el resto de las constantes:
 
         \begin{equation}
            \begin{aligned}
                f_{\lambda\alpha}^{pu} =& \left( K_{InvTr}   \hat{i}_\alpha^{pu} \right) >> D_{InvTr} 
                    -  \left(K_{InvTr} \hat{\lambda}_\alpha^{pu}  \right) >> (15-P_{InvTr}) \\
                    &- \left( K_{\omega} \hat{\omega}^{pu} \hat{\lambda}_\beta^{pu} \right)>> D_{\omega} 
                    +  \left( K_{k1\nu} \nu_\alpha^{pu} \right)>> (15-P_{k1\nu}) \\
                    & +  \left( K_{k2\nu}\nu_\beta^{pu} \right) >> (15-P_{k2\nu})\\
                f_{\lambda\beta}^{pu} =& \left( K_{InvTr} \hat{i}_\beta^{pu} \right) >> (15-_{InvTr})
                    -  \left( K_{InvTr} \hat{\lambda}_\beta^{pu}  \right) >> (15-P_{InvTr}) \\
                    & + \left( K_{\omega} \hat{\omega}^{pu} \hat{\lambda}_\alpha^{pu} \right) >> (15-_{\omega})
                    - \left( K_{k2\nu} \nu_\alpha^{pu} \right) >> (15-P_{k2\nu}) \\
                    & + \left( K_{k1\nu} \nu_\beta^{pu} \right) >> (15-P_{k1\nu}) \\
                f_{i\beta}^{pu} = &  
                    \left(K_a \hat{i}_\beta^{pu}\right) >> (15-P_a)
                    + \left( K_{L_M\_bT_r}  \hat{\lambda}_\beta^{pu}  \right)  >> (15-P_{L_M\_bT_r}) \\
                    & - \left( K_{L_M\_b}  \hat{\omega}^{pu}  \hat{\lambda}_\alpha^{pu}\right)  >> (15-P_{L_M\_b})
                    + \left( K_{L_r\_b}  v_\beta^{pu} \right) >> (15-P_{L_r\_b})
                     -    \nu_\beta^{pu}   \\
                f_{i\alpha}^{pu} = & \left( K_a \hat{i}_\alpha^{pu}        \right) >> (15-P_a)
                   + \left( K_{L_M\_bT_r} \hat{\lambda}_\alpha^{pu}  \right)  >> (15-P_{L_M\_bT_r}) \\
                   & + \left(  K_{L_M\_b} \hat{\omega}^{pu} \hat{\lambda}_\beta^{pu}   \right)>> (15-P_{L_M\_b})
                   + \left( K_{L_r\_b} v_\alpha^{pu} \right) >> (15-P_{L_r\_b} )
                   -  \nu_\alpha^{pu}  \\
                f_\omega^{pu} = & K_\gamma 
                    \left( 
                        l_\alpha \hat{\lambda}_\beta^{pu} 
                        - l_\beta \hat{\lambda}_\alpha^{pu}  
                    \right) 
            \end{aligned}
        \end{equation}

        \begin{equation}
            \begin{aligned}
                \hat{\lambda}_\alpha^{pu}(n) &= \hat{\lambda}_\alpha^{pu}\left(n-1\right)
                    + \left( K_{T_s}  f_{\lambda\alpha}^{pu}\left(n-1\right) \right) >> \left(15-P_{T_s}\right) \\
                \hat{\lambda}_\beta(n) &= \hat{\lambda}_\beta^{pu}\left(n-1\right)
                    + \left(  K_{T_s} f_{\lambda\beta}^{pu} \left(n-1\right) \right) >> \left(15-P_{T_s}\right) \\
                \hat{i}_\beta^{pu}(n) &= \hat{i}_\beta^{pu} \left(n-1\right)
                    +  \left(  K_{T_s}  f_{i\beta}^{pu} \left(n-1\right) \right) >> \left(15-P_{T_s}\right) \\
                \hat{i}_\alpha^{pu}(n) &= \hat{i}_\alpha^{pu} \left(n-1\right) 
                    +  \left( K_{T_s} f_{i\alpha}^{pu} \left(n-1\right) \right) >> \left(15-P_{T_s}\right) \\
                \hat{\omega}^{pu}(n) &= \hat{\omega}^{pu} \left(n-1\right)
                    +  \left(  K_{T_s}  f_{\omega}^{pu}\left(n-1\right)  \right) >> \left(15-P_{T_s}\right)
            \end{aligned}
         \end{equation}

        Las constantes se redefinen como:
        \begin{equation}
            \begin{aligned}
                K_{InvTr}       &= \frac{1}{T_r}  2^{15} \frac{1}{2^{P_{InvTr}}}    &
                K_{k1\nu}       &= k_1 2^{15} \frac{1}{2^{P_{k1\nu} }}              \\
                K_{k2\nu}       &= k_2 2^{15} \frac{1}{2^{P_{k2\nu}  }}             &
                K_{\omega}      &= \omega_0 2^{15}   \frac{1}{2^{P_{\omega}  }}      \\
                K_a             &=    -a 2^{15}     \frac{1}{2^{P_a  }}             &
                K_{L_M\_bT_r}   &= \frac{L_M^2}{b T_r}2^{15} \frac{1}{2^{P_{L_M\_bT_r}  
                    }}  \\
                 K_{L_M\_b}     &=  \frac{L_M^2}{b}2^{15}  \frac{1}{2^{P_{L_M\_b}  }} &
                K_{L_r\_b}      &=  \frac{L_r V_0}{b I_0}2^{15}  \frac{1}{2^{P_{L_r\_b} 
                    }}  \\
                \nu_\beta^{pu}  &= \frac{1}{I_0} \nu_\beta                          &
                \nu_\alpha^{pu} &= \frac{1}{I_0} \nu_\alpha                         \\
                 K_\gamma       &= -\gamma \frac{\lambda_0}{\omega_0}2^{15}
                     \frac{1}{2^{P_\gamma  }}                                       &
                K_{T_s}         &= T_{sc} 2^{15}   \frac{1}{2^{P_{T_s} }}
            \end{aligned}
        \end{equation}
        
        Donde las constantes $P$ se utilizan para ajustar los valores de pre-división.
        


        \subsection{Reporte: Cuantifiación en MATLAB, segunda etapa}
            \begin{description}
            	\item[Fecha]  21/06/2019 
            \end{description}
        
            Se observa que al realizar la cuantificación con el método propuesto anteriormente, el valor de la constante $ K_{T_s} $ queda con muy poca resolución (1 o 2 bits). Se analiza incorporar la constante $T_s$ en las funciones $f_xxx$ para ver si se mejora la resolución de las constantes.


            \subsubsection{Tiempo discreto con constantes, redefinición}
                

		
	%----------------------------------------------------------------------------------------
	%	BIBLIOGRAPHY
	%----------------------------------------------------------------------------------------
	
	\bibliographystyle{apalike}
	
	\bibliography{reposrtes_MI_dspic}
	
	%----------------------------------------------------------------------------------------
	
	
\end{document}